\documentclass[10pt,a4paper]{article}
\usepackage[utf8]{inputenc}
\usepackage{amsmath}
\usepackage{amsfonts}
\usepackage{amssymb}
\author{Malo Kerebel}
\usepackage{cancel}
\begin{document}

\section{Modèle de base}


\paragraph{Navier Stokes }

\begin{equation}
	\rho \dfrac{D \vec{u}}{D t} = - \nabla p + \mu \nabla^2 \vec{u} + \vec{F}
\end{equation}

Modèle tension de surface :
\begin{equation}
	\vec{F} = \rho \vec{g} + \overbrace{\left[ 4 \eta \phi (\phi - 1) \left(phi - \dfrac{1}{2} \right) - k\nabla^2 \phi \right] \nabla \phi}^{Tension}
\end{equation}

\[ \vec{T} = \left[ 4 \eta \phi (\phi - 1) \left(\phi - \dfrac{1}{2} \right) - k\nabla^2 \phi \right] \nabla \phi\]

Avec :
\[\eta = 12 \dfrac{\sigma}{\xi} \quad \quad \quad \kappa = \dfrac{3}{2} \sigma \xi \quad \quad \quad \xi = 3 dx
\]

\paragraph{Bousinessq}

\begin{equation}
	\underbrace{\dfrac{\rho}{\rho_0}}_{\simeq 1} \dfrac{D u}{D t} = \dfrac{-1}{\rho_0} \nabla p + \dfrac{\mu }{\rho_0} \nabla^2 \vec{u} + \dfrac{\vec{F}}{\rho_0}
\end{equation}

\[
	\underbrace{\dfrac{\rho}{\rho_0}}_{\simeq 1} \dfrac{D u}{D t} = \dfrac{-1}{\rho_0} \nabla p + \dfrac{\mu }{\rho_0} \nabla^2 \vec{u} + \dfrac{1}{\rho_0}\left(\vec{T}+\rho \vec{g}\right)
\]

Ajout de la tension :
\[
	-\nabla p + \vec{F} = - \nabla (p + \rho_0 \psi) + \delta \rho \vec{g} + \vec{T}
\]

Équation de la vorticité sans viscosité :

\[
	\nabla \times (3) \Rightarrow \dfrac{D\omega}{D t} =  \dfrac{1}{\rho_0} \nabla \times \left( \delta \rho \vec{g} + \vec{T} \right)
\]

\[
	\dfrac{D\omega}{D t} = \dfrac{1}{\rho_0} \nabla (\delta \rho) \times \vec{g} + \dfrac{1}{\rho_0} \nabla \times \vec{T}
\]

\paragraph{Notre modèle sans viscosité }

\[
	\dfrac{D \vec{\omega}}{D t} = \dfrac{1}{\rho_0} \nabla (\delta \rho) \times \vec{g} + \dfrac{1}{\rho_0} \nabla \times \left( \left[ 4 \eta \phi (\phi_1) \left(\phi - \dfrac{1}{2}\right) - \kappa \nabla^2 \phi \right] \nabla \phi \right)
\]

Simplification de  due à la 2d: 

\[  \dfrac{1}{\rho_0} \nabla (\delta \rho) \times \vec{g} = \dfrac{1}{\rho_0} \partial_x (\delta \rho) \cdot g~\vec{e}_z \]


Donc, ce que l'on intègre en temps :
\[
	\dfrac{D \vec{\omega}}{D t} = \dfrac{1}{\rho_0} \partial_x (\delta \rho ) \cdot g ~\vec{e_z} + \dfrac{1}{\rho_0} \nabla \times \left( \left[ 4 \eta \phi (\phi - 1) \left(\phi - \dfrac{1}{2} \right) - k\nabla^2 \phi \right] \nabla \phi\right)
\]

\paragraph{Pourquoi l'approximation Boussinesq peut-elle fonctionner:}

Terme de pression dans l'équation de base de la vorticité:
\[ -\nabla \times \left( \frac{1}{\rho} \nabla p \right) = -\nabla \left( \frac{1}{\rho} \right) \times \nabla p \] 
Les gradients seront presque alignés, car il y a surpression dans la bulle ou la goutte, donc le produit vectoriel est négligeable.

\textbf{Cependant ci c'était vraiment le cas la goutte ou la bulle ne bougeraient pas.} L'approximation de Boussinesq 
 compense la suppression du terme de pression.
\section{Autres modèles possibles}



Courbure:
\[ \kappa = - \nabla \cdot \left( \frac{\nabla\phi}{\left\| \nabla \phi \right\|} \right) \] 

Tension de surface $\vec{F_{\sigma}}$ :
\[ \vec{F_{\sigma}}= - \frac{\sigma}{\rho} \kappa \nabla \phi \]

Torque:
\[ \nabla \times \vec{F_{\sigma}} = - \sigma\nabla\left( \frac{\kappa}{\rho}\right)\times \nabla \phi \]

? On développe au cas où l'on trouve une simplification de $\nabla(\kappa)$ ?:
\[ =-\sigma \left[ \nabla\left( \frac{1}{\rho}\right)\kappa + \frac{1}{\rho} \nabla(\kappa) \right]\times \nabla\phi \]

Rotationnel de la viscosité si l'on décide d'en ajouter:
\[
\nabla\times (\mu/\rho_0\cdot\nabla^2\vec{u}) = 1/\rho_0 \cdot \left[ \nabla(\mu) \times \nabla^2\vec{u} + \mu \nabla^2\omega \right]
\]





\end{document}